\chapter*{Conclusion}
	\thispagestyle{conclusion}
	\addcontentsline{toc}{chapter}{Conclusion}
	
\par Grâce à la synthèse de code, un développeur peut se concentrer uniquement sur les tests. Le développement de ces derniers n'est donc plus une perte de temps car c'est l'unique chose qu'il doit faire. OLS-Test pourrait permettre au monde professionnel ne pas voir les tests comme une perte de temps, mais plutôt comme une alternative au classique cahier des charges. On peut également imaginer intégrer OLS-Test dans les cycles d'intégration continue. Il suffirait uniquement de faire évoluer les tests, et le code s'adapterait tout seul.

\par OLS-Test est encore jeune et il pourrait être grandement améliorer. Remplacer le lancement en ligne de commande par un plugin pour environnement de développement permettrait d'être plus \textit{user-friendly}. Il subsiste encore également de nombreuses optimisations possibles, comme par exemple compiler les fichiers sans utiliser maven car se dernier est chronophage. Pour finir, il pourrait être intéressant de lancer la synthèses de plusieurs méthodes en parallèle.

\begin{center}
\end{center}
	
	

